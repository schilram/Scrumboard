%----------------------------------------------------------------------------------------
% Instructions:
% Compile using LaTeX and XeLaTeX
%----------------------------------------------------------------------------------------

%----------------------------------------------------------------------------------------
%	PACKAGES AND OTHER DOCUMENT CONFIGURATIONS
%----------------------------------------------------------------------------------------
\documentclass[a4paper,aps,secnumarabic,balancelastpage,nofootinbib]{revtex4}

% Documentclass Options
    % aps, prl stand for American Physical Society and Physical Review Letters respectively
    % twocolumn permits two columns, of course
    % nobalancelastpage doesn't attempt to equalize the lengths of the two columns on the last page
        % as might be desired in a journal where articles follow one another closely
    % amsmath and amssymb are necessary for the subequations environment among others
    % secnumarabic identifies sections by number to aid electronic review and commentary.
    % nofootinbib forces footnotes to occur on the page where they are first referenced
        % and not in the bibliography
    % REVTeX 4 is a set of macro packages designed to be used with LaTeX 2e.
        % REVTeX is well-suited for preparing manuscripts for submission to APS journals.


\usepackage[german]{babel}
\usepackage[babel,german=swiss]{csquotes}
\usepackage{chapterbib}    % allows a bibliography for each chapter (each labguide has it's own)
\usepackage{color}         % produces boxes or entire pages with colored backgrounds
\usepackage{graphics}      % standard graphics specifications
\usepackage[pdftex]{graphicx}      % alternative graphics specifications
\usepackage{longtable}     % helps with long table options
\usepackage{epsf}          % old package handles encapsulated post script issues
\usepackage{bm}            % special 'bold-math' package
%\usepackage{asymptote}     % For typesetting of mathematical illustrations
\usepackage{thumbpdf}
\usepackage{fontspec,xltxtra,xunicode}
\usepackage{hyperref}
	\hypersetup{hidelinks=true}	% Hide boxes arond links
\renewcommand*\familydefault{\sfdefault} %% Only if the base font of the document is to be sans serif


\usepackage[draft=false, debug=false]{hyperref}
\hypersetup{
	pdfauthor={Ramon Schilling},
	pdftitle={ZHAW Web Technologies and Engineering},
	colorlinks,%
    citecolor=black,%
    filecolor=black,%
    linkcolor=black,%
    urlcolor=black
}
\usepackage[paper=a4paper]{geometry}


%\title{\vspace{-1cm}\Large{\sffamily Web Projekt Scrumboard} \Tiny{\author{\sffamily Jan Humbel}}}

\begin{document}

\title{Scrum Board}
\author{Ramon Schilling}
%\date{\today}
\affiliation{ZHAW Studiengang Informatik - Web Technologies and Engineering}
\maketitle

\section{Aufgabe}
Das Ziel des Projekts ist ein Tool zu entwickeln, um ein einfaches Scrum Board verwalten zu können.
Das Scrumboard muss aus den Spalten Todo, In Progress und Done bestehen.

In jeder Spalte muss man Karten (welche Tasks beschreiben) hinzufügen, bearbeiten und löschen können. Ausserdem müssen die Karten zwischen den verschiedenen Spalten verschoben werden können.\\

Jede Karte, bzw. Task muss folgende Eigenschaften aufweisen: Titel, Beschreibung, Aufwandschätzung (1, 2, 3, 5 oder 8), Name der verantwortlichen Person.

\section{verwendete Techologien/Frameworks/Libraries}

\subsection{Backend}
Das Backend wurde mit Node.js und Express wie in der Vorlesung behandelt realisiert.
\begin{description}
\item[Node.js] ist ein serverseitige Platform zum Betrieb von Netzwerkanwendungen, welche auf der JavaScript runtime von Chrome basiert. Die Website von Node.js ist unter \url{http://nodejs.org/} abrufbar.
\item[Express] ist ein Framework für node.js um einfach Webapplikationen zu erstellen. Die Website von Express ist unter \url{http://expressjs.com/} erreichbar.
\end{description}

\subsection{Frontend}
Für das Frontend habe ich auf folgende Frameworks/Libraries zurückgegriffen.
\begin{description}
\item[RequireJS] ist eine Open-Source-Implementierung für asynchrone Moduldefinition in JavaScript. Es dient dazu JavaScript-Dateien nicht beim Öffnen einer Webseite, sondern erst wenn diese benötigt werden zu laden. Zudem enthält es ein Optimierungswerkzeug um minimierte JavaScript-Dateien zu erstellen um das Laden von Webseiten zu beschleunigen. Unter \url{http://requirejs.org} ist die Website von RequireJS verfügbar.
\item[Backbone.js] ist eine JavaScript-Bibliothek mit RESTful-JSON-Schnittstelle. Backbone basiert auf MVC-Pattern.  Unter \url{http://backbonejs.org} ist die Website von Backbone.js abrufbar. Einzige Abhängikeit von Backbone.js ist Underscore.js (\url{http://underscorejs.org}).
\item[Backbone.Validation] ist ein Validierungsplugin für Backbone.js. Es ist unter \url{http://http://thedersen.com/projects/backbone-validation/} erhältlich
\item[Bootstrap] ist ein häufig verwendetes CSS Framework und unter \url{http://http://getbootstrap.com/} erhältlich.
\end{description}

\subsection{Github und Heroku}
Die Applikation ist auf Github unter \url{https://github.com/schilram/Scrumboard} zu finden.\\
Ausserdem ist die Applikaiton auf Heroku deployed und unter \url{http://desolate-atoll-9574.herokuapp.com/} erreichbar.

\end{document}